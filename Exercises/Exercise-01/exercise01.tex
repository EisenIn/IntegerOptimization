\documentclass[11pt,a4paper]{article}
\usepackage{fourier,graphicx} 
\usepackage{amsmath,amsbsy,bm,amssymb,enumerate,mathrsfs}
 
% Encoding settings
\usepackage[utf8]{inputenc} 
\usepackage{utf8math}
\usepackage{anysize} 


\newcommand{\smat}[1]{ \big(\begin{smallmatrix} #1 \end{smallmatrix}\big)}

\DeclareMathOperator{\vol}{vol}
\renewcommand{\epsilon}{\varepsilon}

\renewcommand{\leq}{\leqslant}
\renewcommand{\geq}{\geqslant}
\DeclareMathOperator{\Tr}{Tr}
\DeclareMathOperator{\conv}{conv}
\DeclareMathOperator{\cone}{cone}

%\pagestyle{fancyplain}



\title{Integer Optimization  \\ Problem Set 1 }

%\author{ Friedrich Eisenbrand}
%  \\
%   EPFL\\
%   Switzerland\\
%   {\small \texttt{friedrich.eisenbrand@epfl.ch}}}}
% } 


\date{To be discussed on Feb. 27.}



\begin{document}

\maketitle 




\begin{enumerate} 
\item An employment agency wants to find a job for a set of workers, $W$. The set of available jobs is $E$. The agency knows whether a worker $i ∈ W$ is qualified for job $j ∈ E$. Describe an integer program to find the maximum number of workers being employed in jobs for which they are qualified. \emph{Hint: Use variables $x_{i,j} ∈ \{0,1\}$ indicating whether worker $i$ was assigned job $j$.}

Now, some companies offering jobs, have a maximum number of new hires. For example, the subset $S⊆ E$ of jobs is offered by IBN. IBN can, for budget reasons, only hire $5$ new people. Incorporate these additional  constraints into your integer program. 
\item Recall that for $X ⊆ ℝ^n$ the convex hull of $X$ is

  \begin{displaymath}
      \conv(X) = \left\{ ∑_{i=1}^t μ_i x_i : t ∈ ℕ_+, μ_i ≥0, x_i ∈ X, ∑_{i=1}^t μ_i =1 \right\}.
  \end{displaymath}
Show that, for $A,B ⊆ ℝ^n$, one has $\conv(A ∪B) = \conv\left( \conv(A) ∪ \conv(B) \right)$.  

\item
  Recall that for $X ⊆ ℝ^n$, $X_I = \conv( X ∩ ℤ^n)$. 
  
  Let $\mathscr{C} = \cone(\{c_1,\dots,c_r\})$ with $c_i ∈ℤ^n$ for $1≤i≤r$. Show that $\mathscr{C}_I = \mathscr{C}$.

  \emph{Hint: Write each element of $\mathscr{C}$ as a convex combination of integer multiples of the $c_i$ and $0$}.

\item In this exercise, you can assume that each cone, $\cone (X)$ with $X ⊆ℝ^n$ finite can be written as $P(A,0) = \{ x ∈ℝ^n : Ax≤0\}$ for some matrix $A$ and vice versa.

  Prove the Minkowski-Weyl theorem: A set $P ⊆ ℝ^n$ is a polyhedron if and only if there exist finite sets $X,Y ⊆ ℝ^n$ with $P = X+Y$.

  \emph{Hint: Let $P = P(A,b)$. Consider the polyhedral cone $\mathscr{C} = \{ \smat{x\\λ} ∈ℝ^{n+1}  :Ax ≤ λb, \, λ≥0 \}$. }

\item Let $P = \{ x ∈ ℝ^2 : (1,α)^T x ≤ 0\}$. If $α ∉ ℚ$, then $P_I$ is not a polyhedron.

% \item Let $P ⊆ ℝ^n$ be a rational polyhedron and suppose that $α = \max\{ c^Tx :x∈ℤ^n, \, Ax≤ b\}$ is bounded. Show that $α =  \max\{ c^Tx :x ∈ P_I\}$ holds.


% \item Let $G = (V,E)$ be a complete graph and let $x^* ∈ ℝ^{|E|}_{≥0}$  be given. Show that one can find a \emph{separating} subtour-elimination constraint if there exists one, i.e., a proper  subset $S$ of $V$ with
%   \begin{displaymath}
%     ∑_{e ∈ δ(S)} x^*_e <2. 
%   \end{displaymath}

%   \emph{Hint: Minimum cut}.

\item Consider the triangle $T ⊆ ℝ^2$ with vertices $(0,0), (a,0), (a, γ)$  where $a ∈ ℕ$ and $γ ∈ ℝ_{≥0}$. Show that there exists a set $S⊆ (T ∩ℤ^2)$ with $|S| = O(\log a)$ such that $T_I = \conv(S)$ holds.

\end{enumerate}



%\bibliographystyle{abbrv}
%\bibliography{books,mybib,papers,my_publications}


 
\end{document}





%%% Local Variables:
%%% mode: latex
%%% TeX-master: t
%%% End:

\documentclass[11pt,a4paper]{article}
\usepackage{fourier,graphicx} 
\usepackage{amsmath,amsbsy,bm,amssymb,enumerate,mathrsfs,xspace}
 
% Encoding settings
\usepackage[utf8]{inputenc} 
\usepackage{utf8math}
\usepackage{anysize} 


\newcommand{\smat}[1]{ \big(\begin{smallmatrix} #1 \end{smallmatrix}\big)}

\newcommand{\setR}{\mathbb{R}}
\newcommand{\setZ}{\mathbb{Z}}
\newcommand{\setQ}{\mathbb{Q}}
\newcommand{\setC}{\mathbb{C}}
\newcommand{\setN}{\mathbb{N}}
\newcommand{\wt}[1]{\widetilde{#1}}
\newcommand{\opt}{{\sc 0/1-opt}\xspace}
\newcommand{\aug}{{\sc 0/1-aug}\xspace}
\newcommand{\psep}{{\sc 0/1-psep}\xspace}
\newcommand{\sep}{{\sc 0/1-sep}\xspace}
\newcommand{\fopt}{{\sc 0/1-testopt\xspace} }


\DeclareMathOperator{\vol}{vol}
\renewcommand{\epsilon}{\varepsilon}

\renewcommand{\leq}{\leqslant}
\renewcommand{\geq}{\geqslant}
\DeclareMathOperator{\Tr}{Tr}
\DeclareMathOperator{\conv}{conv}
\DeclareMathOperator{\rank}{rank}


%\pagestyle{fancyplain}



\title{Integer Optimization  \\ Problem Set 3 }

%\author{ Friedrich Eisenbrand}
%  \\
%   EPFL\\
%   Switzerland\\
%   {\small \texttt{friedrich.eisenbrand@epfl.ch}}}}
% } 


\date{To be discussed on March 13}



\begin{document}

\maketitle 




\begin{enumerate} 
\item Let $P,Q ⊆ ℝ^n$ be pointsets.  Show
  \begin{displaymath}
    \conv(P) ⊕ \conv(Q) = \conv(P ⊕Q). 
  \end{displaymath}

\item Pet $P$ and $C$ be  subsets of $ℝ^n$. Show that one has 
  \begin{displaymath}
    P_I ⊕ C_I ⊆ (P ⊕ C)_I.
  \end{displaymath}
  Does equality hold? 
\end{enumerate}

\noindent
The following exercises are deriving the Minkowski-Weyl theorem from \emph{linear programming } theory. % Recall the duality theorem: The linear program $\min\{ c^T y : A y = b, \, y≥0\}$ is feasible and bounded if and only of the linear program $\max\{ b^T x : A^T x ≤ c \}$ is feasible and bounded. In this case, the optimal values of both optimization problems  coincide. 
% We will use this and the separation theorem to show that a finitely generated cone $\cone\{a_1,\dots,a_m\}⊆ ℝ^n$  is polyhedral, i.e. of the form $\{x ∈ ℝ^n : Cx ≤ 0\}$ for some matrix $C ∈ ℝ^{ℓ×n}$.
Recall the following fact from linear programming. If $\max\{ c^T x :x ∈ ℝ^n, \,Ax ≤ b\}$ is feasible and bounded and if $\rank(A) = n$, then there exists an optimal \emph{vertex} solution $x^*$, i.e., $x^*$ satisfies $n$ linear inequalities,  whose left-hand-side vectors are a basis of $ℝ^n$ with equality. 


\begin{enumerate}
  \setcounter{enumi}{2}
\item Let $A ∈ ℝ^{n ×m}$, consider the finitely generated cone $\mathscr{C} = \{ A\,  λ : λ ∈ ℝ^m_{≥0}\}$ and let $b ∉ \mathscr{C} $. With the separation theorem, there exists a hyperplane $x^T z = β$ such that $x^T b > β$ and $x^T z < β$ for each $z ∈  \mathscr{C}$.

  Show that $β≥0$ (on which side is $0$?)  and  $x^T A ≤ 0$ hold.

\item Next consider the linear program
  \begin{displaymath}
     \max\{ b^T x : x ∈ ℝ^n, \,  A^T x ≤ 0, \,  -1 ≤ x ≤ 1\}. 
   \end{displaymath}
   Show that this linear program is feasible and bounded ($b ∉ \mathscr{C}$!) with optimal solution value $>0$ and interpret the optimal solution as a hyperplane, separating $b$ from $\mathscr{C}$. 
 \item Conclude that there is an optimal vertex solution. How many vertex solutions are there at most?
 \item Conclude that $ \mathscr{C}$ is polyhedral, i.e. there exists a matrix $C ∈ ℝ^{m ×n}$ with $\mathscr{C} = \{ Cx ≤0\}$. How does the matrix $C ∈ ℝ^{m ×n}$ look like. How many rows does it have at most. 
   
 \end{enumerate}


\begin{enumerate}
  \setcounter{enumi}{6}
\item Let $\max\{c^Tx : Ax ≤ b, \, x ∈ ℤ^n\}$ be an integer program with $A ∈ℤ^{m × n }$ and $b ∈ℤ^m$. Show that this integer program is unbounded if and only of the linear programming relaxation (The condition $x ∈ ℤ^n$ is replaced by $x ∈ℝ^n$) is unbounded.


\item 
  Solve the following knapsack problem by drawing the corresponding directed acyclic graph:

  \begin{displaymath}
    \max 12x_1 + 10 x_2 + 13 x_3 + 8 x_4 
  \end{displaymath}

  s.t.
  \begin{eqnarray*}
    3x_1 + 5 x_2 + 2 x_3 + 4 x_4 ≤ 10\\
    x_1,x_2,x_3,x_4 ∈ \{0,1\}. 
  \end{eqnarray*}
  
\end{enumerate}

%\bibliographystyle{abbrv}
%\bibliography{books,mybib,papers,my_publications}


 
\end{document}





%%% Local Variables:
%%% mode: latex
%%% TeX-master: t
%%% End:

\documentclass[11pt,a4paper]{article}
\usepackage{fourier,graphicx} 
\usepackage{amsmath,amsbsy,bm,amssymb,enumerate,mathrsfs,xspace}
 
% Encoding settings
\usepackage[utf8]{inputenc} 
\usepackage{utf8math}
\usepackage{anysize} 


\newcommand{\smat}[1]{ \big(\begin{smallmatrix} #1 \end{smallmatrix}\big)}

\newcommand{\setR}{\mathbb{R}}
\newcommand{\setZ}{\mathbb{Z}}
\newcommand{\setQ}{\mathbb{Q}}
\newcommand{\setC}{\mathbb{C}}
\newcommand{\setN}{\mathbb{N}}
\newcommand{\wt}[1]{\widetilde{#1}}
\newcommand{\wb}[1]{\overline{#1}}
\newcommand{\opt}{{\sc 0/1-opt}\xspace}
\newcommand{\aug}{{\sc 0/1-aug}\xspace}
\newcommand{\psep}{{\sc 0/1-psep}\xspace}
\newcommand{\sep}{{\sc 0/1-sep}\xspace}
\newcommand{\fopt}{{\sc 0/1-testopt\xspace} }


\DeclareMathOperator{\vol}{vol}
\renewcommand{\epsilon}{\varepsilon}

\renewcommand{\leq}{\leqslant}
\renewcommand{\geq}{\geqslant}
\DeclareMathOperator{\Tr}{Tr}
\DeclareMathOperator{\conv}{conv}
\DeclareMathOperator{\poly}{poly}
\DeclareMathOperator{\size}{size}
\DeclareMathOperator{\rank}{rank}


%\pagestyle{fancyplain}



\title{Integer Optimization  \\ Problem Set 6 }

%\author{ Friedrich Eisenbrand}
%  \\
%   EPFL\\
%   Switzerland\\
%   {\small \texttt{friedrich.eisenbrand@epfl.ch}}}}
% } 

\date{Working session: April 17, Presentations: April 24}



\begin{document}

\maketitle 

\noindent
Let $Λ ⊆ ℝ^2$ be a lattice and $b_1,b_2 ∈ Λ \setminus \{0\}$ be a basis of $Λ$, ordered such that $\|b_1\|_2 ≤ \|b_2\|_2$. 

\newcounter{carry}
\begin{enumerate}[i)]
  \item Show that $b_1, b_2 -x b_1$, $x ∈ℤ$ is also a basis of $Λ$.
  \item Let $b_2^* = b_2 - μ b_1$ with $μ = 〈b_2,b_1〉 / 〈b_1,b_1〉
    $ be the \emph{projection} of $b_2$ into the \emph{orthogonal
      complement of} $b_1$.

    Prove that, if $|μ| > 1/2$, then $b_2 - ⌊μ⌉ ⋅ b_1$ is strictly shorter than $b_2$, w.r.t. $\|⋅\|_2$. Here $⌊μ⌉$ is the closest integer to $μ$.
  \item If  $b_2 - ⌊μ⌉ ⋅ b_1$ is still longer than $b_1$, then the enclosed angle between these vectors is between $60^\circ$ and $120^\circ$.
    
  \item Show that the following algorithm terminates in $O(\log (\|b_2\|)$ many steps: 
    While $\|b_2^*\| ≤ \frac{1}{4} \|b_1\|$: Replace $b_2$  by  $b_2 - ⌊μ⌉ ⋅ b_1$. Swap $b_1$ and $b_2$.

    \hfill \emph{Hint:  $b_2 - ⌊μ⌉ ⋅ b_1$ is much shorter than $b_2$.}
  \item We call    $b_1,b_2$ \emph{partially reduced} if $\|b_2\|≥ \|b_2\|$ and  $\|b_2^*\| ≥ \frac{1}{4} \|b_1\|$ holds. Show how to compute a shortest nonzero lattice vector in constant time, given a partially reduced basis.

    \hfill \emph{Hint: The length of $x b_1 + y b_2$ is at least $|y| \, \|b_2^*\| ≥ |y| \|b_2\| /  4$.}  
    
\setcounter{carry}{\value{enumi}} 
\end{enumerate}

\noindent 
For a lattice $Λ ⊆ ℝ^n$ and $i ∈ \{1,\dots,n\}$ the number $λ_i(Λ)$ is defined as the minimum $R≥ 0$ such that the ball or radius  $R$ around $0$, $B(0,R) = \{ x ∈ ℝ^n: \|x\|≤ R\} $ contains $i$ \emph{linearly independent} lattice points.

\begin{enumerate}[i)]
  \setcounter{enumi}{\value{carry}} 
\item Show that each lattice $Λ ⊆ ℝ^2$ has a basis $v_1,v_2$ such that $\|v_i\| = λ_i(Λ)$, $i=1,2$  holds.

  \hfill   \emph{Hint: Use the first problems above. This answers the question of Samuel asked in class.}

\item Consider the lattice $    Λ = \{ Ax : x ∈ℤ^5\} $ with $A$ being the matrix
  \begin{displaymath}
    A =
    \begin{pmatrix}
      2 & 0 & 0 & 0 & 1 \\
      0 & 2 & 0 & 0 & 1 \\
      0 & 0 & 2 & 0 & 1 \\
      0 & 0 & 0 & 2 & 1 \\
      0 & 0 & 0 & 0 & 1 \\
    \end{pmatrix}
  \end{displaymath}
  Show that the vectors $2 ⋅e_i$ $i=1,\dots,5$ are attaining the successive minima but do not form a basis of $Λ$.


\item Provide an example of a $2$-dimensional lattice $Λ(b_1,b_2) ⊆ℝ^2$ with $b_1,b_2 ∈ℝ^2$ linearly independent, such that the projection of $Λ$ onto the line generated by $b_1$ is not a (one-dimensional) lattice.   Recall that the  projection of $v$ onto the line generated by $b_1$ is the vector
  \begin{displaymath}
    \frac{〈v,b_1 〉}{ 〈b_1,b_1 〉}   \,  b_1
  \end{displaymath}
\setcounter{carry}{\value{enumi}}





\end{enumerate} 


\noindent
Finally we repeat some basics from Linear Algebra 2. Recall that an integer matrix $U ∈ℤ^{n ×n}$ is called \emph{unimodular} if $\det(U) = \pm 1$ holds. 


\begin{enumerate}[i)]
  \setcounter{enumi}{\value{carry}} 
\item Let $B ∈ ℝ^{n×n}$ be  non-singular and linearly independent and $C ∈ ℝ^{n×n}$.
  One has $Λ(B ) = Λ(C)$ if and only if there exists a unimodular matrix $U ∈ℤ^{n ×n}$ with $B ⋅U = C$.


  Consequently, the absolute value of the determinant of any basis of a lattice $Λ$ is an invariant of the lattice, called the \emph{lattice determinant}, $\det(Λ)$. 



\end{enumerate}

%\bibliographystyle{abbrv}
%\bibliography{books,mybib,papers,my_publications}


 
\end{document}





%%% Local Variables:
%%% mode: latex
%%% TeX-master: t
%%% End:

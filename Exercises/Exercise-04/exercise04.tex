\documentclass[11pt,a4paper]{article}
\usepackage{fourier,graphicx} 
\usepackage{amsmath,amsbsy,bm,amssymb,enumerate,mathrsfs,xspace}
 
% Encoding settings
\usepackage[utf8]{inputenc} 
\usepackage{utf8math}
\usepackage{anysize} 


\newcommand{\smat}[1]{ \big(\begin{smallmatrix} #1 \end{smallmatrix}\big)}

\newcommand{\setR}{\mathbb{R}}
\newcommand{\setZ}{\mathbb{Z}}
\newcommand{\setQ}{\mathbb{Q}}
\newcommand{\setC}{\mathbb{C}}
\newcommand{\setN}{\mathbb{N}}
\newcommand{\wt}[1]{\widetilde{#1}}
\newcommand{\opt}{{\sc 0/1-opt}\xspace}
\newcommand{\aug}{{\sc 0/1-aug}\xspace}
\newcommand{\psep}{{\sc 0/1-psep}\xspace}
\newcommand{\sep}{{\sc 0/1-sep}\xspace}
\newcommand{\fopt}{{\sc 0/1-testopt\xspace} }


\DeclareMathOperator{\vol}{vol}
\renewcommand{\epsilon}{\varepsilon}

\renewcommand{\leq}{\leqslant}
\renewcommand{\geq}{\geqslant}
\DeclareMathOperator{\Tr}{Tr}
\DeclareMathOperator{\conv}{conv}
\DeclareMathOperator{\rank}{rank}


%\pagestyle{fancyplain}



\title{Integer Optimization  \\ Problem Set 4 }

%\author{ Friedrich Eisenbrand}
%  \\
%   EPFL\\
%   Switzerland\\
%   {\small \texttt{friedrich.eisenbrand@epfl.ch}}}}
% } 


\date{To be discussed on March 20 and after}



\begin{document}

\maketitle 

\noindent 
Consider an integer program in so-called \emph{equation standard form}
  \begin{equation}\label{eq:1}
    \max \{ c^T x : Ax = b, \, x≥0, \, x ∈ ℤ^n\}, 
  \end{equation}
  with $A ∈ℤ^{m ×n}$ and $b∈ℤ^m$ and let $Δ$ be an upper bound on the
  absolute value of components of $A$ and $b$. In this exercise, we
  will show that, if the IP is feasible and bounded, then there exists
  an optimal solution $x^*$ with $\|x^*\|_∞ ≤ (m Δ)^{O(m)}$.  To this end,
  recall the \emph{Hadamard bound } $|\det(B)| ≤ ∏_i \|b_i\|_2$ and
  the \emph{matrix inversion formula} $B^{-1} = \wt{B} / \det(B)$,
  where $B ∈ℝ^{m × m}$ is invertible, $\wt{B}$ is the \emph{cofactor
    matrix} (or its transpose), and the $b_i$ are the columns of $B$.

\begin{enumerate} 
\item Let $B ∈ℤ^{m × m}$ be a full-rank matrix with $\|b_{ij}\|_∞ ≤ Δ$
  for each $i,j$. Show that $B^{-1} = C / D$, where
  $0< D ≤ (m ⋅Δ)^{m/2}$ is an integer and $C ∈ ℤ^{m ×m}$ is an integer
  matrix with $|c_{ij}| ≤ (m ⋅Δ)^{m/2}$ for each $i,j$.
\item Let $J ⊆ \{1,\dots,n\}$ be a nonempty index set such that the
  columns $a_j, \, j ∈ J$ are linearly dependent. Show that there
  exists $g ∈ \ker(A) ∩ ℤ^n ⧹\{0\}$ such that
  $\|g\|_∞ ≤ (m Δ)^{2m}$
  and $g_k = 0$ for each $k ∉ J$.\footnote{A generous upper bound.} 
\item
  Let $x^* ∈ℤ^{n}_{≥0}$ be an optimal solution of \eqref{eq:1}. Let $J ⊆ \{1,\dots,n\}$ be the set of indices  with $x^*_j > (m Δ)^{2m}$. Show that the corresponding columns $a_j, \, j ∈ J$ are linearly independent.

  \emph{Hint: This is a bit like in the proof of the Carathéodory Theorem. It can be considered to be an integer version of it. } 

How large can the number of columns $n$ be, if there are no redundant columns?

\item Show that there exists an absolute constant $K$ and an optimal solution $x^*$ that satisfies  $\|x^*\|_∞ ≤ (m Δ)^{K \, m}$. 
\end{enumerate}

\newpage

\noindent 
The next exercise shows how to reduce a problem 
\begin{equation} 
  \label{eq:2}
  A x = b, x ∈ \{0,1\}^n   
\end{equation}
to the solution of one equation
\begin{equation}
  \label{eq:3}
   a^T x = β, x ∈ \{0,1\}^n,   
 \end{equation}
 where $A ∈ℤ^{m ×n}$ $b ∈ℤ^m$, $a ∈ℤ^n$ and $β ∈ℤ$. To simplify the reasoning, we assume that each entry of $A$ is $± 1$ or $0$.
 Let $λ ∈ \{ 0,\dots,M\} ^m$,  $a^T = λ^T A$ and $β = λ^T b$. Furthermore, let $S_1, S_2 ⊆ \{0,1\}^n$ be the set of solutions  of~\eqref{eq:2} and \eqref{eq:3} respectively.

 
 \begin{enumerate}
   \setcounter{enumi}{5}
   \begin{enumerate}
 \item  Show that $S_1 ⊆ S_2$ holds.
 \item An element $x^* ∈ \{0,1\}^n$ belongs to $S_2 ⧹S_1$ if and only if $λ ⊥ (A x^* -b)$.
 \item Suppose that $λ $ is chosen i.i.d. at random from $\{ 0,\dots,M\}$ and that $x^* ∈ \{0,1\}^n ⧹S_1$. Show that
   \begin{displaymath}
     P\left[ λ ⊥ (A x^* -b) \right] ≤ 1/M. 
   \end{displaymath}
   
   \end{enumerate} 
   
 \item 
 \begin{enumerate}
 \item Show that the number of vectors $(A x^* -b), \, x^* ∈\{0,1\}^n$ is bounded by $(2 n +1)^m$.
 \item Show 
   \begin{displaymath}
     P \left[ S_2 ≠ S_2\right] ≤ (2 n +1)^m / M. 
   \end{displaymath}
 \item For $M = (2 n +1)^{2m}$,
   \begin{displaymath}
     P \left[ S_2 ≠ S_2\right] ≤ \frac{1}{2n}^m. 
   \end{displaymath}
   How many bits has the number $M$ then, if it is represented in a computer?
   \end{enumerate}
   \end{enumerate}

 \noindent 
 This is a simple randomized reduction. There are also simple deterministic reductions. Such a reduction shows that the knapsack problem is NP hard. For the following exercise you can assume the existence of such a deterministic reduction with $S_1 = S_2$ guaranteed.
 \begin{enumerate}
   \setcounter{enumi}{12}
\item    Show that the \emph{knapsack problem} is NP hard, by providing a reduction from SAT. 
 \end{enumerate}
 
 

%\bibliographystyle{abbrv}
%\bibliography{books,mybib,papers,my_publications}


 
\end{document}





%%% Local Variables:
%%% mode: latex
%%% TeX-master: t
%%% End:

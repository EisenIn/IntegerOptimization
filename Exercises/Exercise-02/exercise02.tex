\documentclass[11pt,a4paper]{article}
\usepackage{fourier,graphicx} 
\usepackage{amsmath,amsbsy,bm,amssymb,enumerate,mathrsfs,xspace}
 
% Encoding settings
\usepackage[utf8]{inputenc} 
\usepackage{utf8math}
\usepackage{anysize} 


\newcommand{\smat}[1]{ \big(\begin{smallmatrix} #1 \end{smallmatrix}\big)}

\newcommand{\setR}{\mathbb{R}}
\newcommand{\setZ}{\mathbb{Z}}
\newcommand{\setQ}{\mathbb{Q}}
\newcommand{\setC}{\mathbb{C}}
\newcommand{\setN}{\mathbb{N}}
\newcommand{\wt}[1]{\widetilde{#1}}
\newcommand{\opt}{{\sc 0/1-opt}\xspace}
\newcommand{\aug}{{\sc 0/1-aug}\xspace}
\newcommand{\psep}{{\sc 0/1-psep}\xspace}
\newcommand{\sep}{{\sc 0/1-sep}\xspace}
\newcommand{\fopt}{{\sc 0/1-testopt\xspace} }


\DeclareMathOperator{\vol}{vol}
\renewcommand{\epsilon}{\varepsilon}

\renewcommand{\leq}{\leqslant}
\renewcommand{\geq}{\geqslant}
\DeclareMathOperator{\Tr}{Tr}
\DeclareMathOperator{\conv}{conv}
\DeclareMathOperator{\cone}{cone}

%\pagestyle{fancyplain}



\title{Integer Optimization  \\ Problem Set 2 }

%\author{ Friedrich Eisenbrand}
%  \\
%   EPFL\\
%   Switzerland\\
%   {\small \texttt{friedrich.eisenbrand@epfl.ch}}}}
% } 


\date{To be discussed on March 6}



\begin{document}

\maketitle 




\begin{enumerate} 
 \item Show that a polyhedron $P = \{ x ∈ ℝ^n : Ax ≤ b\}$, with $A ∈ ℝ^{m ×n}$ $b∈ ℝ^m$ is a convex set. 
 \item Prove Carath\'eodory's Theorem: Let $X\subseteq\setR^n$, then for each $x \in \cone(X)$ there exists a set
  $\wt{X}\subseteq X$ of cardinality at most $n$  such that $x \in
  \cone(\wt{X})$. The vectors in $\wt{X}$ are linearly independent.
  \item Let $A \in \setR^{n\times n}$ be a non-singular matrix and let
  $a_1,\ldots,a_n\in \setR^n$ be the columns of $A$.  Show that
  $\cone(\{a_1,\ldots,a_n\})$ is the polyhedron $P = \{ y \in \setR^n \colon
  A^{-1} y\geq0\}$. \label{conv:item:3} Show that $\cone(\{a_1,\ldots,a_k\})$ for
  $k\leq n$ is the set $P_k = \{y \in \setR^n \colon
  a_i^{-1} x\geq0, i=1,\ldots,k, \, a_i^{-1}x = 0, i=k+1,\ldots,n\}$, where
  $a_i^{-1}$ denotes the $i$-th row of $A^{-1}$.
\item Prove that for a finite set $X\subseteq\setR^n$ the conic hull $\cone(X)$ is closed  and convex.  Find a countably infinite set $X\subset\setR^2$ such that $\cone(X)$ is not closed.
\item A \emph{vertex} of a polyhedron $P(A,b) ⊆ ℝ^n$  is an element $v ∈ P$ such that there do not exist $v_1, v_2 ∈P$, $v_1 ≠v_2$ with $v = 1/2(v_1+v_2)$. 
  Show that a polyhedron $P(A,b)$ has a vertex if and only if $\mathrm{rank}(A)=n$.
\item Let $P = P(A,b)$ be a rational polyhedron (meaning that both $A$ and $b$ can be chosen to be rational). Show that for each $c ∈ ℝ^n$ one has
  \begin{displaymath}
    \max\{c^Tx : x ∈ P ∩ℤ^n \}  = \max\{c^Tx : x ∈ P_I \}. 
  \end{displaymath}
\item Show that each vertex of the polytope that is described by the inequalities
  \begin{eqnarray*}
    0 ≤ x_i ≤ 1, & i=1,\dots, n \\
    ∑_{i=1}^n x_i ≤ β
  \end{eqnarray*}
  with $β ∈ ℤ$ is an integral point. Argue that the integer program

  \begin{eqnarray*}
    \max ∑_{i=1}^n c_i x_i 
  \end{eqnarray*}

  subject to

  \begin{eqnarray*}
    0 ≤ x_i ≤ 1, & i=1,\dots, n \\
    ∑_{i=1}^n 2 ⋅ x_i ≤ n \\
    x ∈ ℤ^n
  \end{eqnarray*}
 can be solved in polynomial time. 
  
\end{enumerate}



%\bibliographystyle{abbrv}
%\bibliography{books,mybib,papers,my_publications}


 
\end{document}





%%% Local Variables:
%%% mode: latex
%%% TeX-master: t
%%% End:

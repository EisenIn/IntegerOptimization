\documentclass{article}
\usepackage[utf8]{inputenc}
\usepackage[fleqn]{amsmath}
\usepackage{graphicx}
\usepackage{subcaption}
\usepackage{amssymb}
\usepackage{amsthm}
\usepackage{amsfonts}
\usepackage{enumerate}
\usepackage[utf8]{inputenc} 
\usepackage{utf8math}
\usepackage{anysize} 
\usepackage{xcolor}
\usepackage{fourier}
\newcommand{\innerprod}[1]{\left\langle #1 \right\rangle}
\newcommand\norm[1]{\left\lVert#1\right\rVert}
\def\RR{\mathbb{R}}

\title{Integer Optimization \\ Problem Set 8}
\date{Presentations: May 8}

\begin{document}

\maketitle


\begin{enumerate}[i)]
\item 
 Let $\Lambda \subseteq \mathbb{R}^n$ a full rank lattice with basis $b_1, ... , b_n$. A non-zero lattice vector $v$ is said to be primitive if $v$ is not 
 a multiple of any other lattice vector, i.e. $v \neq kw$ for any $w \in \Lambda$ and any $k \in \mathbb{N}_{\geq 2}$. Show that any primitive lattice 
 vector $v$ can be extended to a basis of $\Lambda$, i.e. there are lattice vectors 
 $\tilde{b_2}, ... , \tilde{b_n}$ so that $v, \tilde{b_2}, ... , \tilde{b_n}$ is a basis of $\Lambda$.

 
\emph{Hint: This is a question about unimodular matrices. Write $v = \alpha_1 b_1 + ... + \alpha_n b_n$. Using the Euclidean algorithm, show that there 
exists a unimodular matrix $U \in \mathbb{Z}^{n\times n}$ such that $(\alpha_1, · · · , \alpha_n) \cdot U = (1, 0, ... , 0)$. Observe each operation of the 
Euclidean algorithm only adds / subtracts multiples of some number to / from another number - in the matrix world, this operations corresponds to a 
unimodular matrix. It may be useful to show that gcd$(\alpha_1, ... , \alpha_n) = 1$. Finally, argue that the columns of the matrix $(b_1, b_2, ... , b_n)\cdot U^{-T}$ 
form a basis of $\Lambda$ and its first column is v.}


\item Let $Λ ⊆\mathbb{R}^n$ be a lattice and $u ∈ Λ \setminus \{0\}$ be a shortest vector w.r.t. the $\ell_2$-norm. For $x \in \mathbb{R}^n$ we let
  \begin{displaymath}
    π(x) = x - (x^Tv / v^Tv )\, v
  \end{displaymath}
  be the projection of $x$ and we define $Λ_1 = π(Λ)$ as well as $Λ_1^* = \{ y \in \mathbb{R}^n \colon y \perp v, \, \forall x ∈ Λ_1 \colon y^T x \in \mathbb{Z}\}$.

  Prove or provide a counterexample to the following:
  \begin{displaymath}
    Λ_1^* = π(Λ^*).  
  \end{displaymath}

  \hfill \emph{This is Edwin's question!}


\item 
Let $\Lambda \subseteq \RR^n$ be a full rank lattice. Assume $b_1,... ,b_n \in \Lambda$ are linearly independent and that minimize
$|det(b_1,... ,b_n)|$ over all n linearly independent lattice vectors. Prove that $b_1,... ,b_n$ is a basis of $\Lambda$.

\item 
Let $B \in \mathbb{Q}^{n\times n}$ be a lattice basis that consists of pairwise orthogonal vectors. Prove that the shortest vector of $\Lambda(B)$ is the shortest column vector of B.

\item 
Let $\Lambda \subset \RR^n$ be a lattice. Recall that the dual lattice $\Lambda^\ast$ is defined by $\Lambda^\ast = \{ y \in \mathbb{R}^n : y^Tv \in \mathbb{Z} \quad \forall v \in \Lambda\}$. \\
Let $x \in \RR^d$ a vector. Prove that for every $v \in \Lambda^\ast \backslash \{0\}$ we have that  
$$\frac{\{\langle v, x \rangle \}}{\Vert v \Vert} \leq dist(x, \Lambda)$$
where $\{r\} := |\lceil r \rfloor - r|$ is defined to be the distance from $r$ to the closest integer.
\end{enumerate}

    

\end{document}

%%% Local Variables:
%%% mode: latex
%%% TeX-master: t
%%% End:

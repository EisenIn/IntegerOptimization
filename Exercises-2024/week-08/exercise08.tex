\documentclass[11pt,a4paper]{article}
\usepackage{fourier,graphicx} 
\usepackage{amsmath,amsbsy,bm,amssymb,enumerate,mathrsfs}
 
% Encoding settings
\usepackage[utf8]{inputenc} 
\usepackage{utf8math}
\usepackage{anysize} 


\newcommand{\smat}[1]{ \big(\begin{smallmatrix} #1 \end{smallmatrix}\big)}


\DeclareMathOperator{\vol}{vol}
\DeclareMathOperator{\Spa}{span}

\renewcommand{\epsilon}{\varepsilon}

\renewcommand{\leq}{\leqslant}
\renewcommand{\geq}{\geqslant}
\DeclareMathOperator{\Tr}{Tr}
\DeclareMathOperator{\conv}{conv}
\DeclareMathOperator{\cone}{cone}
\DeclareMathOperator{\RR}{\mathbb{R}}
\DeclareMathOperator{\ZZ}{\mathbb{Z}}
\DeclareMathOperator{\argmin}{\mathrm{argmin}}
\DeclareMathOperator{\CVP}{\mathrm{CVP}}
\DeclareMathOperator{\E}{\mathcal{E}}
% \DeclareMathOperator{\conv}{\mathrm{conv}}
%\DeclareMathOperator{\cone}{\mathrm{cone}}
%\pagestyle{fancyplain}



\title{Integer Optimization  \\ Problem Set 8 }

%\author{ Friedrich Eisenbrand}
%  \\
%   EPFL\\
%   Switzerland\\
%   {\small \texttt{friedrich.eisenbrand@epfl.ch}}}}
% } 


\date{ April 15, 2024}



\begin{document}

\maketitle 


\begin{enumerate}



\item This exercise finishes the argument of the \emph{heart of the proof} of John's theorem. Let
  \begin{displaymath}
    ℰ = \{ x ∈ ℝ^n : ∑_{i=1}^n \frac{1}{α_i^2} (x_i - t_i)^2 ≤ 1\},  
  \end{displaymath}
  where $t = e_1$, $α_1 = 2$, $α_2=\dots=α_n  = \sqrt{n / \left(n+ \frac{1}{10}\right)}$. Show the following. If $x^* ∈ ℰ$ and $x^*_1 ≤0$, then $\|x\|_2 ≤ 1$.  
  

 
\item Consider the \emph{simplex} $Σ = \conv \{e_1,\dots,e_n\} ⊆ ℝ^n$. We interpret $Σ$ as \emph{full-dimensional} in the hyperplane
  \begin{displaymath}
    ∑_i x_i = 1. 
  \end{displaymath}
  
  \begin{enumerate}[a)]
   
  \item Determine the center and radius of the largest euclidean ball contained in $Σ$.
  \item Show that this ball is the largest volume ellipsoid that is contained in $Σ$.
  \end{enumerate}
  
  

\item Consider  the \emph{simplex} $Σ = \conv \{0, n⋅e_1,\dots,n⋅e_n\} ⊆ ℝ^n$.
  \begin{enumerate}[a)]
  \item Show that the \emph{interior} of $Σ$ does not contain an integer point. (I.e. all points in $Σ ∩ℤ^n$ are on the boundary of $Σ$.)
  \item Show that $w(Σ) = n$,
    where $a:=\left(\frac{1}{2},\hdots,\frac{1}{2}\right)$. Show that $K∩\mathbb{Z}^n =\emptyset$. Then give $w(K)$ (up to a constant factor), where $w(K)$ denotes the \emph{lattice width} of $K$
  \begin{displaymath}
    w(K) = \min_{d ∈ ℤ^n \setminus \{0\}} \max_{x,y ∈K} d^T(x - y).
  \end{displaymath}
  
 \end{enumerate}
% \item Consider $K:= \left\{x∈\mathbb{R}^n |∥x−a∥_2 \leq \frac{1}{4}\sqrt{n}\right\}$ where $a:=\left(\frac{1}{2},\hdots,\frac{1}{2}\right)$. Show that $K∩\mathbb{Z}^n =\emptyset$. Then give $w(K)$ (up to a constant factor), where $w(K)$ denotes the \emph{lattice width} of $K$
%   \begin{displaymath}
%     w(K) = \min_{d ∈ ℤ^n \setminus \{0\}} \max_{x,y ∈K} d^T(x - y).
%   \end{displaymath}
  
\item Let $K ⊆\mathbb{R}^n$ be a symmetric convex body. Show that there is an ellipsoid $E$ (with the origin as center) so that $K ⊆ E ⊆ C\sqrt{n}\cdot K$ for some (large) constant $C >0$.


\item Let $∥\cdot∥_∗$ be any norm in $\mathbb{R}^n$. Show that there is a matrix $A$ so that $∥Ax∥_2 \leq ∥x∥_∗ \leq \sqrt{n}∥Ax∥_2$.
\item For $X⊆\mathbb{R}^n$, $X_I=\conv(X∩\mathbb{Z}^n)$.
Let $C=\cone(\{c_1,\hdots,c_r\})$ with $c_i ∈ \mathbb{Z}^n$ for $1\leq i\leq r$. Show that $C_I =C$.

\item In this exercise, you can assume that each cone, $\cone(X)$ with $X ⊆ \mathbb{R}^n$ finite can be written as $$P(A,0)=\{x∈\mathbb{R}^n: Ax\leq 0\}$$ for some matrix A and vice versa. \\
Prove that for $P ⊆ \mathbb{R}^n$, the following statements are equivalent:
\begin{enumerate}
    \item $P$ is a polyhedron, i.e., there exist $A ∈ \mathbb{R}^{d×n}$ and $b ∈ \mathbb{R}^d$ for some $d$ such that
$$P =\{x:Ax\leq b\}$$
\item $P =\conv(\{v_1,\hdots,v_s\})+\cone(\{r_1,\hdots,r_t\})$ for some finitely many vectors $v_1, \hdots, v_s, r_1, \hdots, r_t \in \mathbb{R}^n$. 
\end{enumerate}

\end{enumerate}


%\bibliographystyle{abbrv}
%\bibliography{books,mybib,papers,my_publications}


 
\end{document}





%%% Local Variables:
%%% mode: latex
%%% TeX-master: t
%%% End:

\documentclass[11pt,a4paper]{article}
\usepackage{fourier,graphicx} 
\usepackage{amsmath,amsbsy,bm,amssymb,enumerate,mathrsfs}
 
% Encoding settings
\usepackage[utf8]{inputenc} 
\usepackage{utf8math}
\usepackage{anysize} 


\newcommand{\smat}[1]{ \big(\begin{smallmatrix} #1 \end{smallmatrix}\big)}


\DeclareMathOperator{\vol}{vol}
\DeclareMathOperator{\poly}{poly}
\DeclareMathOperator{\Spa}{span}
\DeclareMathOperator{\interior}{int}


\renewcommand{\epsilon}{\varepsilon}

\renewcommand{\leq}{\leqslant}
\renewcommand{\geq}{\geqslant}
\DeclareMathOperator{\Tr}{Tr}
\DeclareMathOperator{\conv}{conv}
\DeclareMathOperator{\cone}{cone}
\DeclareMathOperator{\RR}{\mathbb{R}}
\DeclareMathOperator{\ZZ}{\mathbb{Z}}
\DeclareMathOperator{\argmin}{\mathrm{argmin}}
\DeclareMathOperator{\CVP}{\mathrm{CVP}}
\DeclareMathOperator{\E}{\mathcal{E}}
\DeclareMathOperator{\V}{\mathcal{V}}
% \DeclareMathOperator{\conv}{\mathrm{conv}}
%\DeclareMathOperator{\cone}{\mathrm{cone}}
%\pagestyle{fancyplain}



\title{Integer Optimization  \\ Problem Set 12 }

%\author{ Friedrich Eisenbrand}
%  \\
%   EPFL\\
%   Switzerland\\
%   {\small \texttt{friedrich.eisenbrand@epfl.ch}}}}
% } 


\date{ May 13, 2024}



\begin{document}

\maketitle 


\begin{enumerate}

\item Let $A ∈ ℤ^{ m ×n}$ be an integer matrix and $b ∈ ℤ^m$ and consider the feasibility problem 
  \begin{equation}
    \label{eq:1}
    Ax = b, \quad x ∈ \{0,1\}^n.  
  \end{equation}
  Let $λ ∈ \{0,\dots,M\}^m$ be chosen i.i.d. at random and consider the feasibility problem 
  \begin{equation}
    \label{eq:2}
    (λ^TA)\,  x = (λ^Tb), \quad x ∈ \{0,1\}^n 
  \end{equation}
  that is defined by one equation only. Let $S_1 ⊆ \{0,1\}$ be the solutions of~\eqref{eq:1} and $S_2 ⊆ \{0,1\}$ be the solutions of~\eqref{eq:2}.
  \begin{enumerate}
  \item Let $x^* ∈ \{0,1\}^n$. Show that $x^* ∈ S_2$ if and only if
    \begin{displaymath}
      λ^T (A x^* - b) = 0.
    \end{displaymath}
  \item Let $v ∈ℤ^n \setminus \{0\}$. Show that
    \begin{displaymath}
      P\left[ λ^T v = 0\right] ≤ 1/M 
    \end{displaymath}  
  \item Let $Δ$ be the largest absolute value of a component of $A$.  Prove that
    \begin{displaymath}
      P \left[ S_2 ≠ S_1\right] ≤ (2 ⋅ n ⋅ Δ +1)^m / M
    \end{displaymath}
\hfill     \emph{Hint: How large is the set $\{Ax - b : x ∈ \{0,1\}^n \}$?} 
  \item Provide a lower bound on $M$ such that
     \begin{displaymath}
      P \left[ S_2 ≠ S_1\right] ≤  1/2
    \end{displaymath}
    holds. 
  \end{enumerate}
\item $A ∈ ℤ^{ m ×n}$ be a nonzero integer matrix and $b ∈ ℤ^m$ and consider the feasibility problem 
  \begin{equation} 
    \label{eq:3}
    Ax = b, \quad x ∈ \{0,1\}^n.  
  \end{equation}
  Show that \eqref{eq:3} is equivalent to a feasibility problem
  \begin{displaymath}
    A'x = b', x≥0, \quad x ∈ ℤ^n.    
  \end{displaymath}
  such that the following conditions hold:
  \begin{enumerate}[i)] 
  \item $A' ∈ℤ^{ (m+n) ×2⋅n}$. 
  \item $\|A' \|_∞ ≤  \|A\|_∞$.
  \item The maximum number of nonzero elements in any row of $A'$ is
    bounded by the maximum of $2$ and the maximum number of nonzero
    elements of any row of $A$.
  \end{enumerate}

\item The \emph{Exponential Time Hypothesis} states that there exists an $ε>0$ such that   $3$-SAT  with $n$ variables cannot be solved in time $2^{ε ⋅n}$. Via the \emph{sparsification lemma} this implies that there exists an $ε>0$ such that   $3$-SAT  with $n$ variables and $O(n)$ clauses cannot be solved in time $2^{ε ⋅n}$.  We will use the latter assumption in this exercise. To this end, let $Φ$ be a $3$-SAT formula with $n$ variables and $O(n)$ clauses. 
  \begin{enumerate}
  \item Describe a feasibility problem of the form
    \begin{displaymath}
      Ax = b, x≥0, \, x ∈ ℤ^n 
    \end{displaymath}
    where $A ∈ \{ \pm 1,0 \}^{O(n) × O(n)}$ that is equivalent to the given satisfiability problem $Φ$. 
  \item  Find an equivalent system
    \begin{displaymath}
      A'x = b', x≥0, \, x ∈ ℤ^n 
    \end{displaymath}
    where $A' ∈ ℤ^{O(n / \log n) × O(n)}$ and $\|A'\|_∞ ≤\poly (n)$.
  \item Show that there cannot be an algorithm that solves the integer feasibility problem
    \begin{displaymath}
      Ax = b, \, x≥ 0, \, x ∈ℤ^n, 
    \end{displaymath}
    where $A ∈ ℤ^{m ×n}$, $\|A\|_∞ ≤Δ$ in time $ (2 ⋅m ⋅Δ)^{m / \log m}$. 
    
  \end{enumerate}
  
  
\item Let $P$ and $Q$ be two simple polygons in $\mathbb{R}^2$, where $P$ has $m$ vertices and $Q$ has $n$ vertices. What is the maximum number of vertices on the boundary of the Minkowski sum $P ⊕ Q$ (asymptotically) assuming:
\begin{enumerate}
    \item $P$ and $Q$ are both convex
    \item $P$ is convex but $Q$ is arbitrary
    \item $P$ and $Q$ are both arbitrary
\end{enumerate}

\item As the leader of an oil-exploration drilling venture, you must determine the least-cost selection of 5 out of 10 possible sites. Label the sites $S_1 , S_2 ,\hdots, S_{10}$ , and the exploration costs associated with each as $C_1 ,C_2 ,\hdots,C_{10}$.
Regional development restrictions are such that:
\begin{enumerate}
    \item Evaluating sites $S_1$ and $S_7$ prevents you from exploring $S_8$.
    \item Evaluating site $S_3$ or $S_4$ prevents you from assessing site $S_5$
    \item Of the sites $S_5, S_6, S_7, S_8$ only two sites may be assessed. 
\end{enumerate}
Formulate an integer program to determine the minimum-cost exploration scheme that satisfies these restrictions.

\item Three different items are to be routed through three machines. Each item must be processed first on machine 1, then
on machine 2, and finally on machine 3. The sequence of items may differ for each machine. Assume that the times
$t_{ij}$ required to perform the work on item $i$ by machine $j$ are known and are integers. Our objective is to minimize the total time necessary to process all the items.
\begin{enumerate}
    \item Formulate the problem as an integer programming problem.
    \item Suppose we want the items to be processed in the same sequence on each machine. Change the formulation accordingly.
\end{enumerate}

\item Let $A$ be a totally unimodular matrix in $\mathbb{R}^{m \times n}$ with rows $A_1, \hdots, A_m$. Let $\delta$ be the maximal value such that for every choice of $n$ rows $A_{i1}, \hdots, A_{in}$ either 
\begin{enumerate}
    \item $A_{in} \in \text{span}(A_{i1}, \hdots, A_{i(n-1)})$ or 
    \item $\text{dist}(A_{in}, \text{span}(A_{i1}, \hdots, A_{i(n-1)})) \geq \delta$. 
\end{enumerate}
Show that $\delta \geq \frac{1}{n}$. 

\item A set $S$ is in isotropic position if for a random point $x ∈ S$, \begin{enumerate}
    \item $\mathbb{E}[x] = 0 $
    \item $\forall v \in \mathbb{R}^n, \mathbb{E}[(v^Tx)^2] = \Vert v \Vert^2$
\end{enumerate}
Prove the following two claims:
\begin{enumerate}[(i)]
    \item $x \in S$ satisfies the above conditions if and only if $\mathbb{E}[xx^T ] = I$.
    \item Any convex set can be put in isotropic position by an affine transformation.
\end{enumerate}



\end{enumerate}


%\bibliographystyle{abbrv}
%\bibliography{books,mybib,papers,my_publications}


 
\end{document}





%%% Local Variables:
%%% mode: latex
%%% TeX-master: t
%%% End:

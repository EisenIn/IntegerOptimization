\documentclass[11pt,a4paper]{article}
\usepackage{fourier,graphicx} 
\usepackage{amsmath,amsbsy,bm,amssymb,enumerate,mathrsfs}
 
% Encoding settings
\usepackage[utf8]{inputenc} 
\usepackage{utf8math}
\usepackage{anysize} 


\newcommand{\smat}[1]{ \big(\begin{smallmatrix} #1 \end{smallmatrix}\big)}


\DeclareMathOperator{\vol}{vol}
\DeclareMathOperator{\Spa}{span}

\renewcommand{\epsilon}{\varepsilon}

\renewcommand{\leq}{\leqslant}
\renewcommand{\geq}{\geqslant}
\DeclareMathOperator{\Tr}{Tr}
\DeclareMathOperator{\conv}{conv}
\DeclareMathOperator{\cone}{cone}
\DeclareMathOperator{\RR}{\mathbb{R}}
\DeclareMathOperator{\ZZ}{\mathbb{Z}}

%\pagestyle{fancyplain}



\title{Integer Optimization  \\ Problem Set 4 }

%\author{ Friedrich Eisenbrand}
%  \\
%   EPFL\\
%   Switzerland\\
%   {\small \texttt{friedrich.eisenbrand@epfl.ch}}}}
% } 


\date{ March 11, 2024}



\begin{document}

\maketitle 




\begin{enumerate}[1)]

\item Consider the integer matrix 
  \begin{displaymath}
A =     \left(\begin{array}{rrr}
1 & 4 & 7 \\
2 & 5 & 8 \\
3 & 6 & 10
\end{array}\right)
\end{displaymath}
and its Gram-Schmidt orthogonalization

\begin{displaymath}
  A = \left(\begin{array}{rrr}
1 & \frac{12}{7} & \frac{1}{6} \\
2 & \frac{3}{7} & -\frac{1}{3} \\
3 & -\frac{6}{7} & \frac{1}{6}
\end{array}\right)
⋅\left(\begin{array}{rrr}
1 & \frac{16}{7} & \frac{53}{14} \\
0 & 1 & \frac{16}{9} \\
0 & 0 & 1
\end{array}\right) 
\end{displaymath}
\begin{enumerate}[i)]
\item What is the matrix in $ℤ^{3 ×3}$ that is the result of the \emph{normalization step} of the LLL-algorithm?
\item Is this result LLL-reduced? If not, which columns should be swapped?
\item Carry on LLL-reduction  with your favorite computer algebra system.

  {\small \emph{I have created this example with sage. It is slightly annoying that Gram-Schmidt is row-wise instead of column-wise. All matrices need to be transposed. You can find the code in the Github-Repository
      {\tt https://github.com/EisenIn/IntegerOptimization} \\
      under the directory\\ 
      {\tt Sage/Exercises-2024/Exercise-04} } }
\end{enumerate}
\item Let $Λ ⊆ ℝ^n$ be a lattice of dimension $k≤n$. In this exercise, you are to construct a orthogonal matrix $U ∈ ℝ^{n × n}$  ($U^T ⋅U = I_n$) such that the first $n-k$ components of  $U ⋅ v$ are zero, for each $v ∈ Λ$. This means that we can identify $Λ$ as a full-dimensional lattice in $ℝ^k$ when it comes to $ℓ_2$-related problems. 

  \begin{enumerate}[i)] 
  \item Explain how to use Gram-Schmidt orthogonalization to find an
    orthonormal basis $v_1,\dots,v_{n-k}∈ ℝ^n$ of the orthogonal
    complement of $Λ$.
  \item Let $b_1,\dots,b_k$ be a basis of $Λ$. Explain how
    Gram-Schmidt on $v_1,\dots,v_{n-k},b_1,\dots,b_k$ (in this order
    and with normalization) constructs a orthogonal matrix
    $U ∈ ℝ^{ n ×n}$ that rotates $v_i$ into $e_i$ respectively, for
    $i=1,\dots,n-k$.
  \item Show that the image of $\{ U ⋅v : v ∈ Λ\}$ is a $k$-dimensional lattice such that the first $n-k$ components of each lattice vector are zero.
  \item Let $Λ' ⊆ ℝ^k$ be the lattice that is obtained from $\{ U v : v ∈ Λ\}$ after deleting the first $n-k$ components. Show that $Λ'$ is full-dimensional and that $\det(Λ') = \det(Λ)$ holds. 
  \end{enumerate}
\item Let $Λ⊆ℝ^n$ be a full-dimensional lattice. In this exercise, we will prove that $Λ$ has a basis $B ∈ℝ^{n ×n}$  that is reduced in the following sense:
  \begin{displaymath}
   ∏_i^n \|b_i\| ≤ 2^{n (n-1)/2}  |\det (B)|.
 \end{displaymath}
 We refer to the notation used in Exercise 1) of Problem Set~2, i.e., let $v ∈Λ \setminus \{0\}$ be a shortest vector of $Λ$ and $Λ'$ the projection of $Λ$ into the orthogonal complement of $v$.
 \begin{enumerate}[i)]
 \item Show that $\dim(Λ') = n-1$.
 \item Moreover 
   \begin{displaymath}
     \det(Λ') = \det(Λ) / \|v\|.  
   \end{displaymath}
 \item Let $b_1', \dots , b_{n-1}'$ be a basis of $Λ'$. One has
   \begin{displaymath}
     \| b_i'\| ^2 + (1/4) \|v\|^2 ≥ \|v\|^2
   \end{displaymath}
   and therefore
   \begin{displaymath}
     \| b_i'\|   ≥ \sqrt{3/4}\|v\| ≥ 1/2 \|v\|. 
   \end{displaymath}
   \hfill{\emph{Hint: Pythagoras!}}
    \item Show that there exists a basis $\{v,b_1,\dots,b_n\}$ of $Λ$ such that
   \begin{displaymath}
     b_i = b_i' + λ_i v ,  \,   i=1,\dots,n-1,
   \end{displaymath}
   where $|λ_i | ≤ 1/2$ for each $i$ and therefore each $b_i$ satisfies
   \begin{displaymath}
     \| b_i\|  ≤  2 \| b_i'\|, \, i=1,\dots,n-1.
   \end{displaymath}

 \item Using induction on the dimension of the lattice, we assume the inequality
   \begin{eqnarray*}
     ∏_{i=1}^{n-1}  \| b_i'\| & ≤ &   2^{(n-1) (n-2)/2}  \det (Λ') \\
                            & =  & 2^{(n-1) (n-2)/2}  \det (Λ) /\|v\|. 
   \end{eqnarray*}
   Conclude
   \begin{displaymath}
     ∏_i^n \|b_i\| ≤ 2^{n (n-1)/2}  |\det (B)|.
   \end{displaymath}



 \end{enumerate}

 

 



\item Show that for any pair of dual bases $B^⊤D = I$, the Gram matrix of the dual
$D^⊤D$ is the inverse of the Gram matrix of the primal $B^⊤B$.

\item Let $B,D ∈ \mathbb{R}^{d×n}$ be a pair of dual bases. For all $i = 1, \hdots,n$, $[b^\ast_i, \hdots, b^\ast_n]$ and $[d_i, \hdots , d_n]$ are also dual bases.

  
\end{enumerate}



%\bibliographystyle{abbrv}
%\bibliography{books,mybib,papers,my_publications}


 
\end{document}





%%% Local Variables:
%%% mode: latex
%%% TeX-master: t
%%% End:

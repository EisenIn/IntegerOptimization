\documentclass[11pt,a4paper]{article}
\usepackage{fourier,graphicx} 
\usepackage{amsmath,amsbsy,bm,amssymb,enumerate,mathrsfs}
 
% Encoding settings
\usepackage[utf8]{inputenc} 
\usepackage{utf8math}
\usepackage{anysize} 


\newcommand{\smat}[1]{ \big(\begin{smallmatrix} #1 \end{smallmatrix}\big)}


\DeclareMathOperator{\vol}{vol}
\DeclareMathOperator{\Spa}{span}
\DeclareMathOperator{\interior}{int}


\renewcommand{\epsilon}{\varepsilon}

\renewcommand{\leq}{\leqslant}
\renewcommand{\geq}{\geqslant}
\DeclareMathOperator{\Tr}{Tr}
\DeclareMathOperator{\conv}{conv}
\DeclareMathOperator{\cone}{cone}
\DeclareMathOperator{\RR}{\mathbb{R}}
\DeclareMathOperator{\ZZ}{\mathbb{Z}}
\DeclareMathOperator{\argmin}{\mathrm{argmin}}
\DeclareMathOperator{\CVP}{\mathrm{CVP}}
\DeclareMathOperator{\E}{\mathcal{E}}
\DeclareMathOperator{\V}{\mathcal{V}}
% \DeclareMathOperator{\conv}{\mathrm{conv}}
%\DeclareMathOperator{\cone}{\mathrm{cone}}
%\pagestyle{fancyplain}



\title{Integer Optimization  \\ Problem Set 11 }

%\author{ Friedrich Eisenbrand}
%  \\
%   EPFL\\
%   Switzerland\\
%   {\small \texttt{friedrich.eisenbrand@epfl.ch}}}}
% } 


\date{ May 6, 2024}



\begin{document}

\maketitle 


\begin{enumerate}

\item Let $K ⊆ ℝ^n$ be a centrally symmetric convex set. Let $u ∈ ℝ^n$ such that
  \begin{equation}
    \label{eq:1}    
    \left( \frac{1}{2} ⋅ K  + u \right) ∩ K ≠ ∅.
  \end{equation}
  \begin{enumerate}
  \item Show that $(1/2) ⋅ K  + u ⊆ 2 ⋅K$.
  \item Let $u_1,\dots,u_k ∈ ℝ^n$ be such that \eqref{eq:1} holds. Furthermore, assume that
    \begin{displaymath}
       \left(\frac{1}{2} ⋅ K  + u_i  \right)∩ \left(\frac{1}{2} ⋅ K  + u_j \right)  = ∅ \text{ for all } i≠j.
         \end{displaymath}
Show that $k ≤ 4^n$.          
\end{enumerate}

\item Let $Λ⊆ℝ^n$ be a full-dimensional lattice with Voronoi cell $\mathcal{V}$ and Voronoi relevant vectors $v_1,\dots,v_{2k} ∈ Λ$ and let $K ⊆ ℝ^n$ be a convex body. Consider the following finite graph $G = (V,E)$ with
  \begin{displaymath}
    V = \left\{ v ∈ Λ : \interior \left((\mathcal{V} + v ) ∩ K \right) ≠ ∅  \right\}
  \end{displaymath}, where $\interior$ refers to the interior 
  and
  \begin{displaymath}
    E = \{ uv : u,v ∈ Λ \text{ and } u-v = v_j \text{ for some } j.  \} 
  \end{displaymath}

  Show that $G$ is connected.
\item Prove or provide a counterexample: Let $Λ ⊆ ℝ^n$ be a full-dimensional lattice,  $t ∈ ℝ^n$ and $A ∈ ℝ^{ n×n}$ be non-singular. Then $v ∈ Λ$ is a closest vector to $t$ if and only if $A v ∈ A⋅Λ$ is a closest vector to $At$ in the lattice $A ⋅Λ$. 
  
\item Let $Λ ⊆ ℝ^n$ be a full-dimensional lattice and $d ∈ Λ^* \setminus \{0\}$ and $β ∈ ℤ$ and $t ∈ ℝ^n$. We consider the problem
  \begin{equation}
    \label{eq:2}
    \min \{ \| v -t \|: v  ∈ Λ, d^T v = β\}. 
  \end{equation}
  \begin{enumerate}
  \item   Describe a target $t^*∈ ℝ^n$  such that this is equivalent to the problem
    \begin{displaymath}
      \min \{ \| v -t^* \|: v  ∈ Λ, d^T v = 0\}.
    \end{displaymath}
  \item Describe a lattice $Λ' ⊆ ℝ^{n-1}$ and a target $t' ∈ ℝ^{n-1}$ such that the corresponding CVP is equivalent to~\eqref{eq:2}. 
  \end{enumerate}

  
  \item Let $a_1 ,\hdots,a_n∈ [0,1]$ be numbers. Show that there exists an $x \in \{-1, 0, +1\}^n \setminus \{0\}$ such that $$\left\vert \displaystyle\sum_{i =1}^n x_i a_i \right\vert \leq \frac{n}{2^n -1}.$$

  \item Let $Λ$ be a lattice and let $Λ_0 ⊂ Λ$ be a sublattice. Prove that $$ρ(Λ) ≤ ρ(Λ_0) ≤ |Λ/Λ_0| \cdot ρ(Λ).$$

  \item Let $P ⊂ \mathbb{R}^d$ be a rational polyhedron and let $v$ be a vertex of $P$. Prove that $v$ has rational coordinates.

  \item Let $P ⊂ \mathbb{R}^d$ be rational polytope. Prove that there exists a positive integer $δ$ such that $δP$ is an integer polytope.

  \item Let $P ⊂ \mathbb{R}^d$ be a rational polyhedron which contains a straight line. Prove that there exists a vector $m ∈ \mathbb{Z}^d \setminus \{0\}$ such that $P + m = P$.

  \item Let $P_1,P_2 ⊂ \mathbb{R}^d$ be polyhedra and let $v_1 ∈ P_1$ and $v_2 ∈ P_2$ be points. Let us define $P_1 × P_2 ⊂ \mathbb{R}^d ⊕ \mathbb{R}^d = \mathbb{R}^{2d}$ by
$$P_1×P_2= \{(x,y): x∈P_1,y∈P_2\}.$$ 
Prove that $P = P_1 × P_2$ is a polyhedron
      
      

\end{enumerate}


%\bibliographystyle{abbrv}
%\bibliography{books,mybib,papers,my_publications}


 
\end{document}





%%% Local Variables:
%%% mode: latex
%%% TeX-master: t
%%% End:

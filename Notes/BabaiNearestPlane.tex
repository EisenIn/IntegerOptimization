\documentclass[11pt,a4paper]{article}
\usepackage{fourier,graphicx} 
\usepackage{amsmath,amsbsy,bm,amssymb,enumerate,mathrsfs}
 
% Encoding settings
\usepackage[utf8]{inputenc} 
\usepackage{utf8math}
\usepackage{anysize} 

\newcommand{\smat}[1]{ \big(\begin{smallmatrix} #1 \end{smallmatrix}\big)}


\DeclareMathOperator{\vol}{vol}
\DeclareMathOperator{\Spa}{span}

\renewcommand{\epsilon}{\varepsilon}

\renewcommand{\leq}{\leqslant}
\renewcommand{\geq}{\geqslant}
\DeclareMathOperator{\Tr}{Tr}
\DeclareMathOperator{\conv}{conv}
\DeclareMathOperator{\cone}{cone}
\DeclareMathOperator{\RR}{\mathbb{R}}
\DeclareMathOperator{\ZZ}{\mathbb{Z}}

%\pagestyle{fancyplain}



\title{Babai's nearest plane algorithm}

%\author{ Friedrich Eisenbrand}
%  \\
%   EPFL\\
%   Switzerland\\
%   {\small \texttt{friedrich.eisenbrand@epfl.ch}}}}
% } 



\date{ today}



\begin{document}

\maketitle
\begin{abstract}
  We complete the proof of the lecture from 29-th of April 2024. The goal is to analyze the LLL-rounding for the closest vector problem. The result is a $2^{O(n)}$-factor approximation. This is also known as Babai's nearest plane algorithm. 
\end{abstract}


The following can be found in~\cite{babai1986lovasz}. 


\bibliographystyle{alpha}
\bibliography{mybib,papers,books}

 
\end{document}





%%% Local Variables:
%%% mode: latex
%%% TeX-master: t
%%% End:

\documentclass[11pt,a4paper]{article} 
\usepackage{fourier,graphicx} 
\usepackage{amsmath,amsbsy,bm,amssymb,enumerate,mathrsfs}
 
% Encoding settings
\usepackage[utf8]{inputenc} 
\usepackage{utf8math}
\usepackage{anysize} 

\newcommand{\smat}[1]{ \big(\begin{smallmatrix} #1 \end{smallmatrix}\big)}


\DeclareMathOperator{\vol}{vol}
\DeclareMathOperator{\Spa}{span}

\renewcommand{\epsilon}{\varepsilon}

\renewcommand{\leq}{\leqslant}
\renewcommand{\geq}{\geqslant}
\DeclareMathOperator{\Tr}{Tr}
\DeclareMathOperator{\conv}{conv}
\DeclareMathOperator{\cone}{cone}
\DeclareMathOperator{\RR}{\mathbb{R}}
\DeclareMathOperator{\ZZ}{\mathbb{Z}}

%\pagestyle{fancyplain}



\title{Babai's nearest plane algorithm}

%\author{ Friedrich Eisenbrand}
%  \\
%   EPFL\\
%   Switzerland\\
%   {\small \texttt{friedrich.eisenbrand@epfl.ch}}}}
% } 



\date{ today}



\begin{document}

\maketitle
\begin{abstract}
  \noindent 
  We complete the proof of the lecture from 29-th of April 2024. The goal is to analyze the LLL-rounding for the closest vector problem. The result is a $2^{O(n)}$-factor approximation. This is also known as Babai's nearest plane algorithm (Babai 1986). 
\end{abstract}



\noindent 
We recall the following. We are given a basis $B ∈ ℚ^{n ×n}$ and a target vector $t ∈ ℚ^n$. The goal is to find a lattice vector $v ∈ Λ(B)$  that is close to $t$. To this end, we suppose that $B$ is LLL-reduced with Gram-Schmidt orthogonalization
\begin{displaymath}
  B = \left( b_1^*,\dots,b_n^* \right) ⋅ R
\end{displaymath}
where $R∈ ℚ^{n × n}$ is un upper triangular matrix with ones on the diagonal. We let $y^* ∈ ℚ^n$ with $t = B y^*$. In class we have argued that we can find a vector $x ∈ℤ^n$ such that
\begin{equation}
  \label{eq:1}
  \| r_1 =  R (x - y^*)\|_∞ ≤ 1/2. 
\end{equation}
Furthermore, if $z ∈ℤ^n$ is a vector different from $x$, then
\begin{equation}
  \label{eq:1}
  \| r_2 = R (z - y^*)\|_∞ ≥ 1/2. 
\end{equation}
In particular, the last component of $r_2$ that differs from $r_1$   is at least $1/2$ in absolute value. Let this be component $i$. We can write
\begin{displaymath}
  r_1 = r_1' + u \text{ and }  r_2 = r_2' + u 
\end{displaymath}
where $u$ is the \emph{common part} of both vectors, i.e., the common components $i+1,\dots,n$.  We want to show 
\begin{displaymath}
 \| B^* r_1  \|^2 ≤ \sqrt{n} 2^{n-1} \| B^* r_2  \|^2. 
\end{displaymath}
With Pythagoras, we have
\begin{displaymath}
  \| B^* r_i  \|^2 =  \| B^* r'_i  \|^2 +  \| B^* u  \|^2  \text{ for }i=1,2. 
\end{displaymath}
Thus it is enough to show 

\begin{displaymath}
 \| B^* r'_1  \|^2 ≤ 2^{n} \| B^* r'_2  \|^2. 
\end{displaymath}
But
\begin{displaymath}
  \| B^* r'_2  \|^2 ≥ \frac{1}{4} \|b_i^*\|^2 
\end{displaymath}
and
\begin{displaymath}
   \| B^* r'_1  \|^2 ≤ \frac{1}{4} ∑_{j=1}^i \|b_j^*\|^2 ≤ n ⋅ 2^{n-1} \|b_i^*\|^2 ,
\end{displaymath}
where then last inequality follows from the LL-reduction criterion $\|b_i^*\|^2 ≤ 2 \|b_{i+1}^*\|^2$.

This implies the guarantee. 


%\bibliographystyle{alpha}
%\bibliography{mybib,papers,books}

 
\end{document}





%%% Local Variables:
%%% mode: latex
%%% TeX-master: t
%%% End:

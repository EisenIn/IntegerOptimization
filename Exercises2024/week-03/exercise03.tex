\documentclass[11pt,a4paper]{article}
\usepackage{fourier,graphicx} 
\usepackage{amsmath,amsbsy,bm,amssymb,enumerate,mathrsfs}
 
% Encoding settings
\usepackage[utf8]{inputenc} 
\usepackage{utf8math}
\usepackage{anysize} 


\newcommand{\smat}[1]{ \big(\begin{smallmatrix} #1 \end{smallmatrix}\big)}


\DeclareMathOperator{\vol}{vol}
\DeclareMathOperator{\Spa}{span}

\renewcommand{\epsilon}{\varepsilon}

\renewcommand{\leq}{\leqslant}
\renewcommand{\geq}{\geqslant}
\DeclareMathOperator{\Tr}{Tr}
\DeclareMathOperator{\conv}{conv}
\DeclareMathOperator{\cone}{cone}
\DeclareMathOperator{\RR}{\mathbb{R}}
\DeclareMathOperator{\ZZ}{\mathbb{Z}}

%\pagestyle{fancyplain}



\title{Integer Optimization  \\ Problem Set 3 }

%\author{ Friedrich Eisenbrand}
%  \\
%   EPFL\\
%   Switzerland\\
%   {\small \texttt{friedrich.eisenbrand@epfl.ch}}}}
% } 


\date{ March 4, 2024}



\begin{document}

\maketitle 




\begin{enumerate} 

\item Find an analogue of Minkowski’s First Theorem for the $l_1$ and $l_∞$ norms.

\item Show that for any full-rank integer lattice $Λ$, $\det(Λ) \cdot \mathbb{Z}_n ⊆ Λ$.


\item For all large enough $n ∈ \mathbb{Z}$, find an $n$-dimensional full-rank lattice in which the successive minima $v_1, \hdots , v_n$ (in the $l_2$ norm) do not form a basis of the lattice. 

Show that for any 2-dimensional full-rank lattice $Λ$, the successive minima $v_1,v_2$ do form a basis of $Λ$.

\item Use Minkowski’s theorem to show the following result of Dirichlet: \\
Let $Q\geq1$ be a real number and let $α_1,\hdots,α_n ∈\mathbb{R}$. There exists an integer $q$ and integers $p_1,\hdots,p_n$ with
\begin{enumerate}
    \item $1 \leq q \leq Q^n$
    \item $|q\cdot \alpha_i - p_i | \leq \frac{1}{Q}$ for $i = 1, \hdots, n$.  
\end{enumerate}

\item Let $Λ := Λ(B) ⊆ \mathbb{R}^n$ be a full rank lattice for the basis $B$. Let $v_1,\hdots,v_n ∈ Λ\backslash \{0\}$ be linear independent vectors. Then $\max\{∥v_1∥_2,\hdots,∥v_n∥_2\} \geq ∥b^∗_n∥_2$ where $b^∗_n$ is the “last” vector in the Gram- Schmidt orthogonalization of $B = (b_1,\hdots,b_n)$.

\item Let $B ∈ \mathbb{R}^{n×n}$ be any regular matrix and let $Λ := Λ(B)$ be the spanned lattice. The orthogonality defect of a lattice is given by 
$$\gamma(B) := \frac{\prod_{i=1}^n \Vert b_i \Vert_2 }{\prod_{i = 1}^n \Vert b^\ast_i \Vert_2 }.$$

Prove that there is a regular matrix $\tilde{B} $ so that $\tilde{Λ}  := Λ(\tilde{B} )$ is a sublattice of $Λ$ (that means $\tilde{Λ}$  is a lattice and $\tilde{Λ}  ⊆ Λ$) and the orthogonality defect satisfies $γ(\tilde{B} ) \leq n^{O(n)}$.

  

\end{enumerate}



%\bibliographystyle{abbrv}
%\bibliography{books,mybib,papers,my_publications}


 
\end{document}





%%% Local Variables:
%%% mode: latex
%%% TeX-master: t
%%% End:

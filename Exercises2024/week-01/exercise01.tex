\documentclass[11pt,a4paper]{article}
\usepackage{fourier,graphicx} 
\usepackage{amsmath,amsbsy,bm,amssymb,enumerate,mathrsfs}
 
% Encoding settings
\usepackage[utf8]{inputenc} 
\usepackage{utf8math}
\usepackage{anysize} 


\newcommand{\smat}[1]{ \big(\begin{smallmatrix} #1 \end{smallmatrix}\big)}


\DeclareMathOperator{\vol}{vol}
\DeclareMathOperator{\Spa}{span}

\renewcommand{\epsilon}{\varepsilon}

\renewcommand{\leq}{\leqslant}
\renewcommand{\geq}{\geqslant}
\DeclareMathOperator{\Tr}{Tr}
\DeclareMathOperator{\conv}{conv}
\DeclareMathOperator{\cone}{cone}

%\pagestyle{fancyplain}



\title{Integer Optimization  \\ Problem Set 1 }

%\author{ Friedrich Eisenbrand}
%  \\
%   EPFL\\
%   Switzerland\\
%   {\small \texttt{friedrich.eisenbrand@epfl.ch}}}}
% } 


\date{ Insert correct date}



\begin{document}

\maketitle 




\begin{enumerate} 
\item

  Let $Λ ⊆ ℝ^n$ be a lattice. A vector $v ∈ Λ$ is called \emph{primitive} if $Λ$ has a basis including $v$.

  \medskip

  Show the following. A lattice vector $v ∈ Λ \setminus \{0\}$ is primitive  if and only if there is no lattice point of the form $μ ⋅ v$ with $0<μ<1$.

\item Show that the set $ \{ x + y \sqrt{2} : x,y ∈ ℤ \} ⊆ ℝ$ is not a lattice.

\item Let $Λ ⊆ ℝ^n$  be a lattice, let $u_1 ,\dots ,u_m  ∈ Λ$  be lattice points and let
  $ L = \Spa \{u_1, \dots , u_m\}$.


  Let us consider the orthogonal projection

  \begin{displaymath}
    \begin{array}{rccc}      
      π : & ℝ^ n &  ⟶ &  ℝ^ n \\
          & x & ↦       & x-l 
          \end{array}
        \end{displaymath}
where $l ∈ L$ such that  $x-l \, ⊥\,  u_i$ for each $i$.         
This is called the projection onto the orthogonal complement $L^⊥$ of $L$.

Prove that the image $Λ_1 = π(Λ)$  is a lattice.
  

  

\end{enumerate}



%\bibliographystyle{abbrv}
%\bibliography{books,mybib,papers,my_publications}


 
\end{document}





%%% Local Variables:
%%% mode: latex
%%% TeX-master: t
%%% End:
